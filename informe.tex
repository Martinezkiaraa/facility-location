% Plantilla de informe de resolución (TP2)

\documentclass[12pt,a4paper]{article}
\usepackage[utf8]{inputenc}
\usepackage[spanish]{babel}
\usepackage{amsmath,amsfonts,amssymb}
\usepackage{graphicx}
\usepackage{hyperref}
\usepackage{geometry}
\usepackage{booktabs}
\usepackage{listings}
\usepackage{caption}
\usepackage{float}
\usepackage{enumitem}
\usepackage{microtype}
\geometry{margin=1in}

% Metadatos (editar según corresponda)
\newcommand{\titulo}{Informe -- TP2}
\newcommand{\autor}{Mariño Martina, Martinez Kiara y Taié Colette}
\newcommand{\asignatura}{Aplicaciones Computacionales en Negocios}

\hypersetup{
	pdftitle={\titulo},
	pdfauthor={\autor},
	colorlinks=true,
	linkcolor=black,
	citecolor=black,
	urlcolor=blue
}

\begin{document}

% Portada
\begin{titlepage}
	\centering
	{\scshape\LARGE \asignatura \par\vspace{0.5em}}
	{\huge\bfseries \titulo \par\vspace{1.5em}}
	{\Large \autor \par\vspace{0.5em}}
\end{titlepage}

% Índices
\tableofcontents
\clearpage

\section{Consigna 1: Asignación de parciales}

\subsection{Definición del problema y datos}

El objetivo es programar los segundos parciales de un conjunto de cursos $P$ en el período comprendido entre el 1/12 y el 12/12, excluyendo los días 6/12, 7/12 y 8/12 (fin de semana y feriado). Para cada día hábil se dispone de cuatro horarios posibles: 9:00, 12:00, 15:00 y 18:00.

Cada parcial $p \in P$ requiere una cantidad $a_p \in \mathbb{Z}_+$ de aulas. Además, se dispone de un grafo de incompatibilidad $G = (P, E)$, donde cada arista $(p,q) \in E$ indica que los parciales $p$ y $q$ tienen estudiantes en común y no pueden tomarse en el mismo día y horario.

En cada día y horario (slot) existen como máximo $A = 75$ aulas disponibles (solo es necesario garantizar que la suma de aulas requeridas nunca supere este límite).

El enunciado también establece que: ``No siempre será posible programar todos los parciales cumpliendo las restricciones, y en ese caso buscamos maximizar la cantidad de parciales programados.''

\subsection{Conjuntos, parámetros y variables}

Trabajamos con los siguientes conjuntos y parámetros:
\begin{itemize}
    \item $P$: conjunto de parciales (índice $p$). En los datos provistos hay en total 208 parciales, identificados como $P0, P1, \dots, P207$.
    \item $D$: conjunto de días hábiles del período de examen:
    \[
      D = \{1, 2, 3, 4, 5, 9, 10, 11, 12\}.
    \]
    \item $T$: conjunto de horarios disponibles:
    \[
      T = \{9, 12, 15, 18\}.
    \]
    \item $a_p \in \mathbb{Z}_+$: cantidad de aulas requeridas por el parcial $p$, leída del archivo \texttt{cursos.dat}.
    \item $A = 75$: cantidad total de aulas disponibles por cada día y horario.
    \item $E \subseteq P \times P$: conjunto de pares incompatibles $(p,q)$, obtenido del archivo 
    
    \texttt{estudiantes-en-comun.dat}. Si $(p,q) \in E$, entonces $p$ y $q$ no pueden coincidir en el mismo día y horario.
\end{itemize}

Definimos la siguiente variable de decisión binaria:
\[
x_{pdt} =
\begin{cases}
1 & \text{si el parcial } p \text{ se programa el día } d \text{ a la hora } t,\\
0 & \text{en caso contrario.}
\end{cases}
\]

\subsection{Modelo de programación lineal entera}

\subsubsection*{Función objetivo}

Buscamos maximizar la cantidad de parciales efectivamente programados. Dado que cada parcial puede aparecer como mucho en un slot (ver Restricción 1), la suma de todas las $x_{pdt}$ cuenta exactamente cuántos parciales fueron asignados:
\begin{equation}
\max \sum_{p \in P} \sum_{d \in D} \sum_{t \in T} x_{pdt}.
\label{consigna1:objetivo}
\end{equation}

\subsubsection*{Restricciones}

\paragraph{(1) A lo sumo un slot por parcial.}

Cada parcial puede ser asignado, como máximo, a un único día y horario:
\begin{equation}
\sum_{d \in D} \sum_{t \in T} x_{pdt} \le 1 
\qquad \forall p \in P.
\label{consigna1:unslot}
\end{equation}

Esta restricción permite que algunos parciales queden sin programar (todas sus $x_{pdt} = 0$), lo cual es coherente con el objetivo de maximizar la cantidad de parciales asignados.

\paragraph{(2) Capacidad de aulas.}

En cada combinación de día y horario, la cantidad total de aulas requeridas por los parciales asignados no puede superar las 75 aulas disponibles:
\begin{equation}
\sum_{p \in P} a_p \, x_{pdt} \le A 
\qquad \forall d \in D,\ \forall t \in T.
\label{consigna1:capacidad}
\end{equation}

\paragraph{(3) Incompatibilidades por alumnos en común.}

Si dos parciales tienen estudiantes en común, no pueden tomarse en el mismo día y horario:
\begin{equation}
x_{pdt} + x_{qdt} \le 1
\qquad \forall (p,q) \in E,\ \forall d \in D,\ \forall t \in T.
\label{consigna1:conflictos}
\end{equation}

\paragraph{(4) Dominio de las variables.}
\begin{equation}
x_{pdt} \in \{0,1\}
\qquad \forall p \in P,\ \forall d \in D,\ \forall t \in T.
\label{consigna1:dominio}
\end{equation}

\subsection{Implementación en ZIMPL y uso de los datos}

El modelo se implementó en ZIMPL de la siguiente manera (fragmento relevante):
\begin{lstlisting}[language=,basicstyle=\ttfamily\small]
set P := {read "cursos.dat" as "<1s>"};
set D := {1, 2, 3, 4, 5, 9, 10, 11, 12};
set T := {9, 12, 15, 18};

param a[P] := read "cursos.dat" as "<1s> 2n";
param A_S := 75;

set E := {read "estudiantes-en-comun.dat" as "<1s,2s>"};

var X[P * D * T] binary;

maximize parciales_asignados:
    sum <p,d,t> in P*D*T: X[p,d,t];

subto UnSlot:
    forall <p> in P do
        sum <d,t> in D*T: X[p,d,t] <= 1;

subto Capacidad:
    forall <d,t> in D*T do
        sum <p> in P: a[p] * X[p,d,t] <= A_S;

subto Conflicto:
    forall <p,q> in E do
        forall <d,t> in D*T do
            X[p,d,t] + X[q,d,t] <= 1;
\end{lstlisting}

\subsection{Resultados y conclusiones}

A partir de la salida obtenida, se observaron los siguientes resultados:
\begin{itemize}
    \item En los datos provistos hay un total de 208 parciales. El modelo logró asignar los 208 parciales a algún día y horario factible. Es decir, la solución óptima alcanza el máximo posible de la función objetivo \eqref{consigna1:objetivo}.
    \item La condición de incompatibilidad (3) se respeta en todos los casos: ningún par de cursos con estudiantes en común fue programado en el mismo día y horario.
    \item La restricción de capacidad (2) también se verifica. Al calcular la suma de aulas $\sum_{p} a_p x_{pdt}$ para cada slot $(d,t)$, el valor máximo resultó ser de 67 aulas, estrictamente por debajo del límite de 75.
\end{itemize}

La distribución de parciales por día se resume en la Tabla~\ref{tab:distribucion-parciales-dia}:
\begin{table}[H]
    \centering
    \begin{tabular}{cc}
        \toprule
        Día & Cantidad de parciales asignados \\
        \midrule
         1 & 47 \\
         2 & 25 \\
         3 & 11 \\
         4 & 16 \\
         5 & 11 \\
         9 & 35 \\
        10 & 30 \\
        11 &  5 \\
        12 & 28 \\
        \bottomrule
    \end{tabular}
    \caption{Distribución de parciales programados por día.}
    \label{tab:distribucion-parciales-dia}
\end{table}

A partir de esta distribución se observa que:
\begin{itemize}
    \item La solución utiliza prácticamente todos los slots disponibles (35 de los 36 posibles), lo cual es esperable dado que se busca maximizar la cantidad de parciales programados.
    \item Existe cierta concentración de parciales en los días 1, 9, 10 y 12, mientras que el día 11 tiene una carga relativamente menor. Esta asimetría es coherente con la función objetivo elegida, que sólo maximiza el número de parciales asignados y no incluye criterios de equidad en la distribución temporal.
\end{itemize}

En resumen, el modelo de programación lineal entera propuesto resulta suficiente para:
\begin{enumerate}
    \item Representar adecuadamente las restricciones de incompatibilidad y capacidad del sistema.
    \item Encontrar una solución en la que se programan todos los parciales dentro del horizonte de planificación y con las aulas disponibles.
\end{enumerate}

\section{Consigna 2: Extensión con restricción adicional}

En esta consigna se agrega una nueva condición: queremos impedir que haya estudiantes rindiendo tres parciales el mismo día. En términos del grafo de incompatibilidad $G = (P,E)$, el enunciado lo expresa como:
``no queremos que haya tres vértices vecinos en $G$ programados un mismo día''

Es decir, si tres parciales $p, q, r \in P$ forman un triángulo en $G$ (los tres pares $(p,q)$, $(p,r)$ y $(q,r)$ pertenecen a $E$), entonces no se permite que los tres estén programados el mismo día, aun cuando estén en horarios distintos.

Como en la consigna 1, trabajamos sólo con información de cursos y su grafo de incompatibilidad; no tenemos información de estudiantes individuales, de modo que debemos modelar esta restricción a partir de la estructura del grafo.

\subsection{Modificación del modelo}

La idea es mantener:
\begin{itemize}
    \item la misma función objetivo (maximizar la cantidad de parciales programados),
    \item las restricciones de un único slot por parcial,
    \item las restricciones de capacidad de aulas,
    \item y las restricciones de incompatibilidad por slot,
\end{itemize}
y sumar una familia adicional de restricciones que impida que los tríos de cursos completamente incompatibles (triángulos en $G$) se programen todos en el mismo día.

\subsubsection*{Tríos completamente incompatibles}

Primero, definimos el conjunto de tríos $(p,q,r)$ tales que todos los pares son incompatibles:
\[
A = \Big\{ (p,q,r) \in P^3 :
\ p \neq q,\ p \neq r,\ q \neq r,\ 
(p,q)\in E,\ (p,r)\in E,\ (q,r)\in E \Big\}.
\]

En ZIMPL, esto se implementó como:
\begin{lstlisting}[basicstyle=\ttfamily\small]
set A := {
    <p,q,r> in P * P * P
    with
        p != q and p != r and q != r
        and <p,q> in E
        and <p,r> in E
        and <q,r> in E
};
\end{lstlisting}

Cada trío en $A$ representa tres parciales que están mutuamente conectados en el grafo de incompatibilidad.

\subsubsection*{Nueva restricción: no tres del mismo trío en el mismo día}

Recordemos que, para un parcial $p$ y un día $d$, la suma:
\[
\sum_{t \in T} x_{pdt}
\]
indica si el parcial $p$ se programa (en algún horario) el día $d$.

La restricción que agregamos es, entonces:
\begin{equation}
\sum_{t \in T} x_{pdt}
+ \sum_{t \in T} x_{qdt}
+ \sum_{t \in T} x_{rdt}
\;\le\; 2
\qquad \forall (p,q,r)\in A,\ \forall d \in D.
\label{consigna2:trioconflicto}
\end{equation}

Esta desigualdad asegura que, para cualquier trío completamente incompatible $(p,q,r)$, a lo sumo dos de esos parciales pueden ser programados el mismo día (en cualquiera de los cuatro horarios disponibles). Dicho de otra forma: ningún estudiante que curse esas tres materias podría tener los tres parciales el mismo día.

En ZIMPL, la restricción se implementó como:
\begin{lstlisting}[basicstyle=\ttfamily\small]
subto TrioConflicto:
  forall <p,q,r> in A do
    forall <d> in D do
      (sum <t> in T: X[p,d,t])
    + (sum <t> in T: X[q,d,t])
    + (sum <t> in T: X[r,d,t]) <= 2;
\end{lstlisting}

\subsection{Modelo completo modificado}

El resto del modelo se mantiene igual que en la Consigna 1:
\begin{itemize}
    \item Función objetivo:
    \[
      \max \sum_{p \in P} \sum_{d \in D} \sum_{t \in T} x_{pdt}.
    \]
    \item Un único slot por parcial:
    \[
      \sum_{d \in D} \sum_{t \in T} x_{pdt} \le 1 \quad \forall p \in P.
    \]
    \item Capacidad de aulas por día y horario:
    \[
      \sum_{p \in P} a_p x_{pdt} \le 75 \quad \forall d \in D,\ \forall t \in T.
    \]
    \item Incompatibilidades por slot:
    \[
      x_{pdt} + x_{qdt} \le 1 \quad \forall (p,q)\in E,\ \forall d \in D,\ \forall t \in T.
    \]
    \item Dominio:
    \[
      x_{pdt} \in \{0,1\} \quad \forall p,d,t.
    \]
    \item Nueva restricción de tríos (Ec.~\eqref{consigna2:trioconflicto}).
\end{itemize}

\subsection{¿Se puede resolver en tiempos de cómputo razonables?}

Desde el punto de vista teórico, el agregado de las restricciones de tríos:
\begin{itemize}
    \item incrementa el número de restricciones del modelo (hay una desigualdad por cada trío en $A$ y por cada día $d \in D$),
    \item y, por lo tanto, vuelve el problema más grande y potencialmente más costoso de resolver.
\end{itemize}

Sin embargo, en la instancia concreta de este trabajo:
\begin{itemize}
    \item el número total de parciales es moderado (208 cursos),
    \item el grafo de incompatibilidad no es completo (no todos los cursos están conectados entre sí),
    \item y el solver \texttt{SCIP} fue capaz de resolver el modelo extendido en tiempos de cómputo razonables, sin problemas de memoria ni tiempos excesivos.
\end{itemize}

\subsection{Impacto sobre la cantidad máxima de parciales programados}

Al volver a resolver el modelo con la nueva familia de restricciones, observamos que:
\begin{itemize}
    \item se siguen pudiendo programar los 208 parciales dentro del período y con las 75 aulas disponibles por slot;
    \item es decir, la nueva restricción \eqref{consigna2:trioconflicto} no reduce la cantidad máxima de parciales que pueden asignarse.
\end{itemize}

Lo que sí cambia es la distribución temporal de los parciales. La solución óptima del modelo extendido concentra más exámenes en los primeros días (1 al 5) y utiliza mucho menos los días 9, 10, 11 y 12.

La Tabla~\ref{tab:distribucion-parciales-dia-consigna2} muestra la cantidad de parciales programados por día con la restricción de tríos:
\begin{table}[H]
    \centering
    \begin{tabular}{cc}
        \toprule
        Día & Cantidad de parciales asignados \\
        \midrule
         1 & 59 \\
         2 & 46 \\
         3 & 30 \\
         4 & 27 \\
         5 & 21 \\
         9 & 11 \\
        10 &  9 \\
        11 &  4 \\
        12 &  1 \\
        \bottomrule
    \end{tabular}
    \caption{Distribución de parciales programados por día con la restricción de tríos.}
    \label{tab:distribucion-parciales-dia-consigna2}
\end{table}

Comparado con la distribución obtenida en la Consigna 1, se observa:
\begin{itemize}
    \item un uso mucho más intensivo de los días 1 a 5,
    \item y una utilización muy baja de los días finales (en particular, el día 12 casi no se usa).
\end{itemize}

Esta reasignación es coherente con el nuevo requisito: el modelo debe evitar que los tríos de cursos completamente incompatibles caigan el mismo día, pero al mismo tiempo intenta mantener el máximo número de parciales asignados. El solver encuentra una forma de reorganizar los exámenes en el calendario respetando todas las restricciones (capacidad, incompatibilidades por slot y tríos) sin sacrificar la cantidad total de parciales programados.

En conclusión:
\begin{enumerate}
    \item El modelo se modifica agregando el conjunto de tríos $A$ y las restricciones de la forma \eqref{consigna2:trioconflicto}.
    \item El problema sigue siendo resoluble en tiempos de cómputo razonables para la instancia analizada.
    \item La cantidad máxima de parciales programados no cambia (se siguen asignando los 208), aunque la distribución de los exámenes a lo largo de los días se ve afectada por la nueva restricción.
\end{enumerate}

\section{Comparación entre Consigna 1 y Consigna 2}

En las secciones anteriores vimos que tanto el modelo base (Consigna 1) como el modelo extendido con tríos incompatibles (Consigna 2) permiten programar los 208 parciales dentro del período disponible y respetando la capacidad de aulas. 

La diferencia entre ambos enfoques no está en la factibilidad global del problema, sino en cómo se distribuyen los exámenes a lo largo de los días. En esta sección resumimos esa diferencia.

\subsection{Distribución de parciales por día}

La Tabla~\ref{tab:comparacion-consigna1-consigna2} muestra la cantidad de exámenes programados en cada día del período considerado, comparando ambos modelos.
\begin{table}[H]
    \centering
    \begin{tabular}{ccc}
        \toprule
        Día & Consigna 1 & Consigna 2 \\
        \midrule
         1 & 47 & 59 \\
         2 & 25 & 46 \\
         3 & 11 & 30 \\
         4 & 16 & 27 \\
         5 & 11 & 21 \\
         9 & 35 & 11 \\
        10 & 30 &  9 \\
        11 &  5 &  4 \\
        12 & 28 &  1 \\
        \bottomrule
    \end{tabular}
    \caption{Comparación de parciales programados por día entre ambos modelos.}
    \label{tab:comparacion-consigna1-consigna2}
\end{table}

\subsection{Visualización comparativa}

En la Figura~\ref{fig:comparacion-parciales} se presenta un gráfico de barras que resume visualmente las diferencias entre ambas soluciones.
\begin{figure}[H]
    \centering
    \includegraphics[width=0.85\textwidth]{comparacion_1vs2.png}
    \caption{Parciales programados por día: comparación entre Consigna 1 y Consigna 2.}
    \label{fig:comparacion-parciales}
\end{figure}

\subsection{Análisis}

Los datos muestran que, aunque ambos modelos asignan los mismos 208 parciales, la restricción adicional de tríos tiene un impacto notable en la distribución temporal de los exámenes:
\begin{itemize}
    \item Los días 1 a 5 concentran ahora una mayor cantidad de parciales en la Consigna 2.
    \item Los días 9, 10 y 12, que en la Consigna 1 tenían una carga sustancial, pasan a tener una utilización muy baja (en particular, el día 12 casi no se usa).
\end{itemize}

En síntesis, la restricción sobre los tríos no altera la viabilidad del problema, pero sí cambia de forma importante la estructura del calendario: se redistribuyen exámenes hacia los primeros días para cumplir con las nuevas condiciones sin sacrificar la cantidad total de parciales programados.

\section{Consigna 3: Maximización de la dispersión entre parciales}

En esta consigna se introduce un nuevo objetivo: maximizar la dispersión temporal de los parciales entre los estudiantes. La motivación es evitar que un estudiante que cursa varias materias tenga todos sus exámenes concentrados en pocos días consecutivos.

El desafío principal es que el enunciado aclara que \emph{no contamos con los datos individuales de los estudiantes}. Sólo conocemos, para cada par de cursos $(p,q)$, la cantidad de estudiantes en común $I_{pq}$. Por lo tanto, cualquier modelo posible debe construirse únicamente a partir de la estructura del grafo de incompatibilidades y de estos pesos.
Para simplificar la complejidad de búsqueda de soluciones, decidimos redefinir el significado de "dispersión". No vamos a buscar cualquier tipo de separación entre parciales que comparten alumnos entre sí, sino más bien una cantidad de días de separación pactada con anterioridad. Así,
podemos transformar el problema en uno mucho más manejable, ya que solo necesitamos verificar si la distancia temporal entre dos parciales cumple con el umbral establecido.

\subsection{Idea general para construir una función objetivo}

Si dos materias $p$ y $q$ tienen muchos estudiantes en común ($I_{pq}$ grande), entonces es deseable que sus parciales se programen lo más separados posible en el calendario. Si se asignan al mismo día o a días consecutivos, los estudiantes compartidos enfrentan una carga excesiva en un período muy corto.

Con esta información limitada, la estrategia razonable consiste en:
\begin{itemize}
    \item asignar un ``beneficio'' proporcional a la distancia en días entre dos parciales;
    \item ponderar ese beneficio por la cantidad de estudiantes en común $I_{pq}$;
    \item y formar una función objetivo que premie asignar lejos y castigue asignar cerca.
\end{itemize}

Esto lleva naturalmente a una expresión del tipo:
\[
\max \sum_{(p,q)\in E} \sum_{d_1\in D} \sum_{d_2\in D}
I_{pq} \cdot |d_1 - d_2| \cdot Z_{pq,d_1,d_2},
\]
donde $Z_{pq,d_1,d_2}=1$ si el curso $p$ está asignado al día $d_1$ y $q$ al día $d_2$.

\subsection{Modelado con variables auxiliares}

Como la función objetivo depende de la distancia entre los días asignados, es necesario introducir la variable auxiliar:
\[
Z_{pq,d_1,d_2} =
\begin{cases}
1 & \text{si } p \text{ está en el día } d_1 \text{ y } q \text{ en el día } d_2, \\
0 & \text{en otro caso}.
\end{cases}
\]

Esta variable se vincula con $X_{pdt}$ mediante restricciones del tipo:
\[
Z_{pq,d_1,d_2} \le \sum_{t\in T} X_{p,d_1,t},
\qquad
Z_{pq,d_1,d_2} \le \sum_{t\in T} X_{q,d_2,t},
\]
\[
Z_{pq,d_1,d_2} \ge \sum_{t\in T} X_{p,d_1,t}
              + \sum_{t\in T} X_{q,d_2,t} - 1.
\]

Estas restricciones fuerzan que $Z$ sea igual a 1 sólo cuando ambos cursos están asignados exactamente a esos días.

\subsection{Dos enfoques propuestos}

En el trabajo se experimentó con dos variantes del modelo:
\begin{enumerate}
    \item Versión 1: evitar mismos días y días consecutivos.\\
    Además de la maximización de dispersión, se agrega la restricción de que dos cursos con estudiantes en común no puedan rendir:
    \begin{itemize}
        \item el mismo día;
        \item ni dos días consecutivos.
    \end{itemize}

    \item Versión 2: extender la distancia mínima entre días.\\
    Se prohíbe que dos cursos ligados por una arista del grafo se tomen con 0, 1 o incluso 2 días de diferencia. Esto fuerza una separación aún mayor y favorece una dispersión más agresiva.
\end{enumerate}

Ambas versiones mantienen todas las restricciones previas: capacidad de aulas, asignación única por parcial y consistencia de las variables.

\subsection{Resultados obtenidos}
\subsubsection{Distribución comparada de parciales por día}

La Tabla~\ref{tab:comparacion-ej3} muestra cuántos parciales quedaron programados en cada día en ambas versiones.
\begin{table}[H]
    \centering
    \begin{tabular}{ccc}
        \toprule
        Día & Versión 1 & Versión 2 \\
        \midrule
         1 & 30 & 27 \\
         2 &  1 &  0 \\
         3 & 24 &  0 \\
         4 &  0 & 27 \\
         5 & 19 & 18 \\
         9 & 24 & 18 \\
        11 & 16 &  0 \\
        12 &  0 & 22 \\
        \bottomrule
    \end{tabular}
    \caption{Cantidad de parciales programados por día en ambas versiones del modelo de dispersión.}
    \label{tab:comparacion-ej3}
\end{table}

\begin{figure}[H]
    \centering
    \includegraphics[width=0.85\textwidth]{comparacion_3_versiones.png}
    \caption{Ejercicio 3: comparación de parciales por día entre la Versión 1 y la Versión 2.}
    \label{fig:comparacion-ej3}
\end{figure}

El análisis muestra diferencias marcadas entre ambas versiones:
\begin{itemize}
    \item La Versión 1 asigna parciales en más días, aunque con fuertes concentraciones (por ejemplo, días 1, 3 y 9), y programa en total 114 parciales.
    \item La Versión 2 asigna menos días pero con separaciones aún mayores, concentrando los exámenes en pocos días muy alejados entre sí y reduciendo la cantidad total a 94 parciales.
\end{itemize}

En ambos casos, la cantidad de parciales programados es muy inferior a los 208 que se lograban en las consignas 1 y 2. Esto cuantifica de forma clara el trade-off discutido previamente: cuanto más se fuerza la dispersión temporal entre materias con estudiantes en común, menor es el número de exámenes que puede programarse dentro de un calendario acotado.\\

\textbf{Parciales asignados}\\
En ambas variantes se observa un fenómeno importante:
\begin{itemize}
    \item La función objetivo ya no busca maximizar la cantidad de parciales asignados, sino la dispersión ponderada por estudiantes en común. 
    \item Como consecuencia, la solución puede dejar muchos cursos sin asignar o agruparlos únicamente en los días que permiten mayor separación relativa.
\end{itemize}

En los resultados presentados, las dos versiones asignaron únicamente una fracción de los 208 parciales. Estas se tratan solo de soluciones parciales en ejecución por 3hs. La distribución se concentra fuertemente en grupos muy separados entre sí, lo cual es coherente con el objetivo de maximizar $|d_1 - d_2|$.  

Es decir: el modelo prefirió \emph{asignar menos parciales} pero lograr una dispersión muy alta entre los que comparten estudiantes.

En particular, al sumar la cantidad de parciales asignados por día se obtiene que:
\begin{itemize}
    \item La Versión 1 programa en total 114 parciales.
    \item La Versión 2 programa en total 94 parciales.
\end{itemize}

Ambos valores son sensiblemente menores a los 208 parciales que se lograban programar en las consignas 1 y 2, lo que cuantifica el costo de imponer una dispersión temporal muy estricta.\\

\textbf{Dispersión}\\
Por otro lado, para comparar la dispersión que se obtuvo en ambas versiones analizamos los valores objetivo obtenidos:
\begin{itemize}
    \item La Versión 1 encontró una valor objetivo de 25069.
    \item La Versión 2 encontró una valor objetivo de 23183.
\end{itemize}
Debido a la función objetivo definida, si dos cursos comparten pocos alumnos, que estén juntos o separados no mueve mucho la FO.
En cambio, si dos cursos comparten muchos alumnos, entonces separarlos en días le da un aporte grande a la FO.
Si bien la Versión 2 impone más dispersión mínima como restricción dura, esto no implica un mayor valor de la función objetivo. Nuestra FO mide la dispersión total ponderada, es decir, 
la suma para todos los pares de cursos con alumnos en común de: \[I_{pq} \cdot |d_1 - d_2|\]
Al endurecer demasiado las restricciones, la Versión 2:
\begin{itemize}
    \item Vuelve más difícil asignar parciales, ya que no existen suficientes días disponibles para cumplir todas las separaciones mínimas exigidas.
    \item Obliga al modelo a dejar cursos sin programar para no violar las restricciones.
    \item Disminuye la cantidad de pares (p,q) que entran en la suma de la función objetivo.
\end{itemize}
Como resultado, aunque los pocos pares que logran asignarse en la Versión 2 pueden estar muy separados, el número total de pares contribuyendo a la dispersión cae drásticamente, y eso 
hace que el valor global de la FO sea menor que en la Versión 1.\\

\section{Conclusión}
De esta comparación sacamos la conclusión de que hay un trade-off entre:
\begin{itemize}
    \item Imponer la dispersión como restricción, lo que vuelve muy difícil encontrar asignaciones factibles y reduce la cantidad de exámenes asignados.
    \item Tratar la dispersión solamente como criterio de optimización, lo cual permite que el modelo tenga libertad para asignar los parciales y luego optimizar su distribución temporal.
\end{itemize}
Entonces, maximizar la dispersión suele entrar en conflicto directo con maximizar la cantidad de exámenes programados.


\end{document}