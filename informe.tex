
% Plantilla de informe de resolución (TP2)
% Archivo: informe.tex
% Descripción: plantilla en español para completar el informe de resolución
% Edítala para añadir tu contenido específico del PDF `tp2.2025.pdf`.

\documentclass[12pt,a4paper]{article}
\usepackage[utf8]{inputenc}
\usepackage[T1]{fontenc}
\usepackage[spanish]{babel}
\usepackage{amsmath,amsfonts,amssymb}
\usepackage{graphicx}
\usepackage{hyperref}
\usepackage{geometry}
\usepackage{booktabs}
\usepackage{listings}
\usepackage{caption}
\usepackage{float}
\usepackage{enumitem}
\usepackage{microtype}
\geometry{margin=1in}

% Metadatos (editar según corresponda)
\newcommand{\titulo}{Informe de resolución -- TP2}
\newcommand{\autor}{Nombre del autor}
\newcommand{\asignatura}{Asignatura / Curso}
\newcommand{\profesor}{Profesor}
\newcommand{\grupo}{Grupo / Matrícula}
\newcommand{\fechaentrega}{\today}

\hypersetup{
	pdftitle={\titulo},
	pdfauthor={\autor},
	colorlinks=true,
	linkcolor=blue,
	citecolor=blue,
	urlcolor=blue
}

\begin{document}

% Portada
\begin{titlepage}
	\centering
	{\scshape\LARGE \asignatura \\\vspace{0.5em}}
	{\huge\bfseries \titulo \\\vspace{1.5em}}
	{\Large \autor \\\vspace{0.5em}}
	{\large \profesor \\\vspace{0.5em}}
	{\large \grupo \\\vspace{2em}}
	{\large Fecha: \fechaentrega \\\vfill}
	\vfill
\end{titlepage}

% Resumen/Abstract
\begin{abstract}
	Escribe aquí un resumen breve (150–250 palabras) que describa el objetivo del problema, la estrategia de resolución y los resultados principales obtenidos a partir del archivo \texttt{tp2.2025.pdf}.
\end{abstract}

	ableofcontents
\listoffigures
\listoftables
\clearpage

\section{Introducción}
Explica el objetivo del informe, el contexto del trabajo práctico y una descripción general de lo que el lector encontrará en el documento.

\section{Enunciado del problema}
\label{sec:enunciado}
Incluye aquí el enunciado extraído de \texttt{tp2.2025.pdf}. Si lo prefieres, añade una transcripción o un resumen con los datos relevantes (entradas, salidas, restricciones y métricas de evaluación).

Ejemplo de lista de elementos a incluir:
\begin{itemize}
	\item Descripción de los datos de entrada
	\item Parámetros y supuestos
	\item Objetivo de la resolución
	\item Criterios de evaluación
\end{itemize}

\section{Metodología}
Describe la estrategia seguida para resolver el problema: algoritmos, modelos, reducción a problemas conocidos, heurísticas y justificación de la elección.

\subsection{Modelado matemático}
Plantea las variables, funciones objetivo, restricciones, y la formulación matemática usada (si aplica).

\subsection{Algoritmo / implementación}
Explica el algoritmo paso a paso. Si incluyes pseudocódigo, usa un entorno de enumeración o \texttt{listings} para fragmentos de código.

\begin{lstlisting}[language=Python, caption={Ejemplo de pseudocódigo/implementación}]
def resolver(datos):
		# Implementación
		pass
\end{lstlisting}

\section{Resolución}
Detalla la resolución propia: pasos realizados, cálculos intermedios, tablas y figuras que muestren resultados parciales.

\subsection{Cálculos y demostraciones}
Incluye aquí las fórmulas, justificaciones y cualquier demostración necesaria.

\subsection{Resultados intermedios}
Presenta tablas y gráficas con los resultados relevantes. Usa \verb|\begin{table}| y \verb|\includegraphics| según convenga.

\section{Resultados finales}
Resume los resultados finales: tablas, métricas y una interpretación clara de los mismos. Señala si los resultados cumplen con los requisitos del enunciado.

\section{Discusión}
Comenta limitaciones, posibles mejoras, alternativas de solución y análisis de complejidad (si procede).

\section{Conclusiones}
Resumir los hallazgos principales y destacar conclusiones prácticas y académicas.

\section*{Agradecimientos}
Opcional: menciona personas o recursos que ayudaron.

\appendix
\section{Anexos}
\subsection{Archivos entregables}
Lista de archivos que acompañan el informe (por ejemplo: \texttt{resolucion.zpl}, código fuente, datos):
\begin{itemize}
	\item resolucion.zpl -- Instrucciones/impresión ZPL (si aplica)
	\item src/ -- Código fuente (si lo incluyes)
	\item datos/ -- Datos usados
\end{itemize}

\subsection{Códigos relevantes}
Incluye fragmentos de código más largos si es necesario o referencia a archivos en el repositorio.

\section{Referencias}
Incluye aquí las referencias bibliográficas en el formato que prefieras (puedes añadir \texttt{bibtex} si lo deseas).

\bigskip
\noindent\textbf{Nota:} Completa cada sección con el contenido específico extraído de \texttt{tp2.2025.pdf} y tus soluciones. Esta plantilla está pensada para que la adaptes cómodamente.

\clearpage
\section*{Instrucciones de compilación}
\addcontentsline{toc}{section}{Instrucciones de compilación}
Compilar con \LaTeX{} en Windows (PowerShell):
\begin{itemize}
	\item Opción rápida (dos pasadas):
		\begin{quote}
			pdflatex -interaction=nonstopmode informe.tex; pdflatex -interaction=nonstopmode informe.tex
		\end{quote}
	\item Opción recomendada si tienes \texttt{latexmk} instalado:
		\begin{quote}
			latexmk -pdf -pdflatex="pdflatex -interaction=nonstopmode" informe.tex
		\end{quote}
	\item Si usas bibliografía (BibTeX):
		\begin{quote}
			pdflatex informe.tex; bibtex informe; pdflatex informe.tex; pdflatex informe.tex
		\end{quote}
\end{itemize}

\vfill
\end{document}

