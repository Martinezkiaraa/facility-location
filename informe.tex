% Plantilla de informe de resolución (TP2)

\documentclass[12pt,a4paper]{article}
\usepackage[utf8]{inputenc}
\usepackage[spanish]{babel}
\usepackage{amsmath,amsfonts,amssymb}
\usepackage{graphicx}
\usepackage{hyperref}
\usepackage{geometry}
\usepackage{booktabs}
\usepackage{listings}
\usepackage{caption}
\usepackage{float}
\usepackage{enumitem}
\usepackage{microtype}
\geometry{margin=1in}

% Metadatos (editar según corresponda)
\newcommand{\titulo}{Informe de resolución -- TP2}
\newcommand{\autor}{Mariño Martina, Martinez Kiara, Taié Colette}
\newcommand{\asignatura}{Aplicaciones Computacionales en Negocios}

\hypersetup{
	pdftitle={\titulo},
	pdfauthor={\autor},
	colorlinks=true,
	linkcolor=blue,
	citecolor=blue,
	urlcolor=blue
}

\begin{document}

% Portada
\begin{titlepage}
	\centering
	{\scshape\LARGE \asignatura \par\vspace{0.5em}}
	{\huge\bfseries \titulo \par\vspace{1.5em}}
	{\Large \autor \par\vspace{0.5em}}
\end{titlepage}

% Índices
\tableofcontents
\listoffigures
\listoftables
\clearpage

\section{Introducción}
El presente informe corresponde al segundo trabajo práctico de la materia \textit{Aplicaciones Computacionales en Negocios}. El objetivo general es aplicar los conceptos de programación lineal y entera para modelar y resolver problemas de optimización vinculados a la gestión de recursos, planificación y toma de decisiones en entornos empresariales.
A lo largo del informe se presentan los distintos modelos desarrollados, comenzando por formulaciones lineales y extendiéndose a versiones con variables enteras y binarias. Cada modelo se formula matemáticamente, se implementa en el lenguaje \texttt{ZIMPL} y se resuelve mediante el solver \texttt{SCIP}, analizando los resultados obtenidos y las decisiones óptimas que surgen de cada caso.
El documento está estructurado en secciones que describen los modelos planteados, los resultados de las simulaciones y una discusión final sobre las conclusiones y aprendizajes obtenidos del trabajo.

\section{Consigna 1: Asignación de parciales}
\subsection{Definición de conjuntos, parámetros y variables}
Para comenzar, definiremos los conjuntos con los que vamos a trabajar a lo largo del proyecto:
\begin{itemize}
    \item $P$: Conjunto de parciales (índices p), que interpretaremos como un equivalente a conjunto de cursos (dado que solo hay un parcial por curso).
    \item $D$: Conjunto de días disponibles para asignar parciales, donde $D = (1, 2, 3, \dots, 12) - (6, 7, 8)$
    \item $T$: Horarios disponibles por días, donde $T = (9, 12, 15, 18)$ (índices t)
		\item $S$: Conjunto de slots, donde $S = D \times T$
    \item $a_p \in Z_+$: Número de aulas que requiere el parcial $p \ \ \forall \ \ p \in P$ 
    \item $A$: Aulas totales disponibles en cada slot ($A = 75$)
    \item $E$: Conjunto de pares incompatibles $(p, q)$. Si $(p, q) \in E \rightarrow $ no pueden coincidir en el mismo slot.
\end{itemize}

	Ahora, tendremos la variable indicadora $x_{p,s} \in \{0,1\}$ que nos señala si un parcial $p$ está asignado al slot $s$. $x_{p,s} = 1$ si el parcial se programa en el slot $s = (d, t)$, y 0 en caso contrario.

\subsection{Función objetivo y restricciones}
\textbf{Función objetivo:} maximizar la cantidad de parciales asignados.
\begin{equation}
\max \sum_{p \in P} \sum_{s \in S} x_{p,s}
\end{equation}
Sujeto a:
\begin{align}
\\text{(1)} \quad & \sum_{s \in S} x_{p,s} \le 1 && \forall p \in P \\
\\text{(2)} \quad & \sum_{p \in P} a_p \, x_{p,s} \le 75 && \forall s \in S \\
\\text{(3)} \quad & x_{p,s} + x_{q,s} \le 1 && \forall (p,q) \in E, \, \forall s \in S \\
\\text{(4)} \quad & x_{p,s} \in \{0,1\} && \forall p \in P, \, \forall s \in S
\end{align}

\section{Consigna 2: Extensión con restricción adicional}
\section{Consigna 3: Maximización de la separación temporal}
\section{Resultados generales y discusión}
\section{Conclusiones}

\end{document}